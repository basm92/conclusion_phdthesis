\documentclass[12pt]{extarticle}
\usepackage[utf8]{inputenc}
\usepackage[margin=1in]{geometry}

\usepackage[cmintegrals,cmbraces]{newtxmath}
%\usepackage{ebgaramond}

\usepackage{amsmath}
\usepackage{amsfonts}
%\usepackage{amssymb}

%table numbers within section
\usepackage{chngcntr}
\counterwithin{table}{section}

\usepackage{natbib}
\usepackage{graphicx}
\usepackage{booktabs}
\usepackage{longtable}
\usepackage{caption}

%\frenchspacing
\usepackage{setspace}
\setstretch{1.1}

\usepackage{graphicx}
\usepackage{url}
\usepackage[colorlinks=true, urlcolor = orange, linkcolor=blue, citecolor=purple]{hyperref}

\usepackage{xcolor}
%\usepackage{sectsty}
\usepackage{appendix}

%\subsectionfont{\color{cyan!70!blue}}  % sets colour of chapters
%\sectionfont{\color{blue!60!black}}  % sets colour of sections

%Tables:
\usepackage{dcolumn}
\usepackage{float}
\usepackage{caption}
\usepackage{tabularx}

% Landscape mode
%\usepackage{lscape}
\usepackage{pdflscape}

%Threepart table
\usepackage{threeparttable}
%\usepackage{tabu}

% Modelsummary
\usepackage{booktabs}
\usepackage{siunitx}
\newcolumntype{d}{S[input-symbols = ()]}

% figure notes
\usepackage[capposition=top]{floatrow}

\title{\textbf{PhD Thesis Conclusion}}
\author{Bas Machielsen \\ Utrecht University}
\date{\today}

\begin{document}

\maketitle

\section{Conclusion}

\subsection{Concluding Remarks}

In [Year X], Dutch Lower House member and liberal politician Sam van Houten accused his fellow Lower House members of being an oligarchy, purposefully shirking their responsibility of contributing a fair share towards the state finances\footnote{I want to thank Jesper van der Most for bringing this quote to my attention} \citep{van2013eerste}. His outburst marked the frustration experienced particularly by the liberal wing of the Lower House and society in general with the failure of the political elite to come to timely and appropriate solutions of the country's most important problems, in this case, the reform of the fiscal system. 


\begin{center}
    [Connect quote with central questions from intro]
\end{center}

This dissertation started out by asking the question if, how, and to what extent a political elite prioritizes its self-interest, and how this phenomenon interrelates with democratization and a transitioning economy. 



\subsection{Recapitulation of the Main Chapters}
The second chapter of this thesis started out by mapping out the wealth of Dutch politicians over time, in various representative bodies. In doing so, the chapter is supplementing the historical literature about the profile of Dutch politicians in the nineteenth and twentieth centuries \citep{van1983toegang, secker1991ministers, van1999eerste, oomen2020werk} with a perspective that was ruled out before: \cite{van1983toegang} remarked that "the hypothesis that involves background factors having a predictive value for political behavior and political choices especially has turned out to be unprovable so far."\footnote{I again thank Jesper van der Most for bringing this quote to my attention.} This chapter can be seen as bringing new life to this way of thinking, making use of a a publicly available source, the \textit{Memories van Successie}, to get a detailed impression of politicians' personal wealth. This chapter laid bare a particularly large cleavage between the Lower and Upper Houses: whereas the lower house was a diverse place in terms of personal wealth, the upper house was exclusively dominated by extremely wealthy individuals, and this remained so throughout the period. Importantly, the data in the second chapter paved the way for asking analytical questions about the consequences of this wealth profile of the representative bodies, which are the subjects of the third and fourth chapters. 

In a sense, the third and fourth chapter in this dissertation can be interpreted as a test of perspectives by e.g. \cite{tahoun2019personal} and \cite{grossman1996electoral}, who think that politicians have an indirect utility function of the form:

\begin{equation}\label{eq:utility}
    V(p) = \gamma (p) + \alpha U (p)
\end{equation}

where $\gamma (p)$ represents the utility from monetary benefits (costs) of adopting a policy $p$, and $U (p)$ represents the utility to the politician by all other factors, for example, ideology, party discipline, or the general interest. The relative weight of these priorities is represented by $\alpha$. In the third chapter, my results show that the personal wealth of politicians influences their decision-making: richer politicians are more likely to vote against fiscal legislation than poorer politicians, everything else equal. This analysis exploits the progressivity of fiscal project laws, and finds that politicians who would be hit harder by the acceptance of a new piece of fiscal legislation are systematically less likely to accept the law than politicians who are poorer, conditional on party and a host of other controls. The instrumental variable analyses also provide support for a causal interpretation of these results. 

I also analyze laws in which the effect of acceptance on personal wealth is not so clear \textit{a priori}, specifically, the case of suffrage extension. Additionally, I conduct various placebo tests, and find there is no effect between the acceptance of the law and the personal wealth of politicians, making it unlikely that personal wealth interests actually proxy something else. In sum, this shows that the personal profile of politicians impacted the acceptance of fiscal legislation, and thus, government size and the level of social spending. In terms of the utility function \ref{eq:utility}, I thus find that the $\gamma (p)$ component is large enough to be detectable statistically. The coefficient on wealth is about five times as small as the coefficient on political party, the most important empirical determinant of voting behavior, allowing me to guess that the parameter $\alpha$ should be somewhere in the neighborhood of five. 

But if self-interest plays an important role in the choices of politicians, why have these laws still been accepted, to the detriment of their own interests? My interpretation makes use of the findings in chapter two and chapter three. In chapter two, I documented the wealth of politicians, and I also focused on the wealth of parliament over time. I observed a pattern of declining parliamentary wealth, which, over the entire period, meant that the median Lower House member was almost 10 times as poor in 1920 than the median Lower House member in 1870. In chapter three, I documented that the speed with which these key laws have been accepted accelerated in the 1900s and 1910s, whereas before, many predecessor law projects with roughly the same goal were rejected by parliament \citep{smit2002omwille, van2013eerste}. I think that there are two main factors underlying these patterns: the first is that certain exogenous economic shocks, which made politicians substantially poorer, have sufficiently mitigated the personal wealth-incentive for these laws to be accepted. Secondly, a part of this effect might have been exacerbated by the franchise extensions we have seen, which increased the diversity of parliament, including in wealth \citep{van1983toegang}. Thus, while there is no evidence of the influence of self-interested behavior in accepting suffrage extensions, these suffrage extensions themselves might have had an indirect effect on mitigating the role of self-interest in politics, through the selection of poorer politicians on average.

This consideration also brings me to the economic relevance of self-interest. The counterfactual scenarios constructed in chapter three imply that, for accepted laws to be rejected, politicians would have had to be richer by a factor of about ten, which is approximately the difference between the wealth of the median Lower House politician in 1870 (before the majority of reforms) and 1920 (afterwards). Conversely, for rejected laws around 1870 to be accepted, politicians would have had to be about 10 times poorer. This implies that the Lower House in 1914, had it been as wealthy as the Lower House in 1870, would likely have rejected the laws they have in fact accepted. Thus, coming back to the quote by \cite{van1983toegang} earlier, I interpret this as evidence of 'background factors', specifically personal wealth, playing a direct role in determining politicians' voting behavior. 

In the fourth chapter, my analysis shows that there are significant returns to politics in the first period, equivalent to approx. a 5 percentage point yearly return on wealth over the remaining lifetime, and also equivalent to about 5-6 a yearly Ministers' salary. The results also show that there are little to no returns to a longer stay in politics, suggesting that the returns to politics are depletable. I rule out various alternative explanations, including the possibility that returns are obtained after a political career, for example, by a lucrative function in finance or in the colonies. 

My interpretation of these results focuses on the nature of the district system \citep{van2018tussen}, promoting close connections between (enfranchised) business leaders and the political representatives of the district, and politicians being able to use their discretion in the lower house to accept law projects that are financially favorable to them, and reject laws that are not. I also provide evidence of the influence of changing institutions on the returns to politics: in particular, I find that these returns are realized in periods when political parties did not exist \citep[cf.][]{eggers2009mps}. This is additional evidence for the supposition that politicians used their discretion to their own financial advantage, hinting again at the existence of the first component in equation \ref{eq:utility}: after political parties were founded, party discipline decreased the level of autonomy and discretion of individual politicians \citep{de2001antirevolutionaire,de2014ons}. According to this analysis, political parties, then, are able to discipline politicians enough to make them refrain from engaging in self-interested activities. On the other hand, several other changing institutional changes, such as suffrage extensions, have not been able to influence the returns to politics \citep[see also][]{ashworth2010does}. 

\subsection{Reflection}

What are the most important insights?

What do we learn about the Netherlands?

What do we learn about theoretical models?

What do we learn about methods? Possibilities and limitations

What do we learn about democratization in general?


\subsection{Suggestions for Further Research}


Politicians turn out to use their discretion to prioritize their own finances when voting on laws, and politicians are able to profit financially from holding office. 

How can politicians obtain financial advantage exactly? Is it by voting favorable laws? Through projects, or is it due to asset returns? Are these opportunities depletable?

Take into account politicians' personal profiles when modeling political economy: intra political-elite heterogeneity.

Strategic voting behavior? How do politicians behave in parliament? Do they reveal their preferences truthfully?

Also, theoretically, it is interesting to look at the interaction between political party demands and politicians' own position: what determines the strength of party discipline.

What determines political party choice? This is also guided by self-interest, presumably. Secondly, what determines the party line? Is it influential ideologically-minded individuals, or is there also room for self-interest?


Policy relevance: how to discipline politicians? More insight into personal finances might help if electorates consider it important enough. 

In the present-day Dutch context, party discipline has almost become absolute. Is there then no more room for abuse by politicians, and is the problem anachronistic? No, because now more subtle ways to prioritize self-interest. Plus, in many other countries, still possible because party discipline is looser. Analyze amendments rather than laws?



\clearpage
\bibliographystyle{apalike}
\bibliography{references}

\end{document}
