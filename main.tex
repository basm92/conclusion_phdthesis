\documentclass[12pt]{extarticle}
\usepackage[utf8]{inputenc}
\usepackage[margin=1in]{geometry}

\usepackage[cmintegrals,cmbraces]{newtxmath}
%\usepackage{ebgaramond}

\usepackage{amsmath}
\usepackage{amsfonts}
%\usepackage{amssymb}

%table numbers within section
\usepackage{chngcntr}
\counterwithin{table}{section}

\usepackage{natbib}
\usepackage{graphicx}
\usepackage{booktabs}
\usepackage{longtable}
\usepackage{caption}

%\frenchspacing
\usepackage{setspace}
\setstretch{1.1}

\usepackage{graphicx}
\usepackage{url}
\usepackage[colorlinks=true, urlcolor = orange, linkcolor=blue, citecolor=purple]{hyperref}

\usepackage{xcolor}
%\usepackage{sectsty}
\usepackage{appendix}

%\subsectionfont{\color{cyan!70!blue}}  % sets colour of chapters
%\sectionfont{\color{blue!60!black}}  % sets colour of sections

%Tables:
\usepackage{dcolumn}
\usepackage{float}
\usepackage{caption}
\usepackage{tabularx}

% Landscape mode
%\usepackage{lscape}
\usepackage{pdflscape}

%Threepart table
\usepackage{threeparttable}
%\usepackage{tabu}

% Modelsummary
\usepackage{booktabs}
\usepackage{siunitx}
\newcolumntype{d}{S[input-symbols = ()]}

% figure notes
\usepackage[capposition=top]{floatrow}

\title{\textbf{PhD Thesis Conclusion}}
\author{Bas Machielsen \\ Utrecht University}
\date{\today}

\begin{document}

\maketitle

\section{Conclusion}

\subsection{Concluding Remarks}

In [Year X], Dutch Lower House member and liberal politician Sam van Houten accused his fellow Lower House members of being an oligarchy, purposefully shirking their responsibility of contributing a fair share towards the state finances \citep{van2013eerste}. His outburst marked the frustration experienced particularly by the liberal wing of the Lower House and society in general with the collective failure of the political elite to come to timely and appropriate solutions of the country's most important problems, in this case, the reform of the fiscal system. 

Van Houten's claim can be decomposed into two aspects: first, he implied that his colleagues acted according to self-interest, attempting to preserve their own positions, rather than acting in the common interest of society. Second, that this has important consequences not only for the political system, but also for wider society. On the one hand, it would not be a surprise if Van Houten was right. \cite{smit2002omwille} documents a very long and protracted struggle around fiscal legislation, which first came to the surface in the 1870's, but was only resolved in 1893, twenty years later, with the establishment of an income tax. The Dutch income tax realized a more equal distribution of taxation, and spread the tax burden more evenly, and likely improved efficiency and social welfare. On the other hand, there might be numerous other reasons why politicians have decided to reject or delay the acceptance of laws, none of which might have something to do with self-interest. 

This dissertation can be seen as an attempt to unveil whether the conjectures of Van Houten were correct, and if so, what the consequences are of politicians pursuing their self-interest. In various chapters, I have analyzed the Dutch political elite in various aspects: I have outlined their personal interests by focusing on one of the most obvious objectives they would want to pursue: increasing their own wealth \citep{buchanan1989essays}. In addition, I have asked whether the persuasion of self-interests is apparent in political decision-making, and if so, what that would imply for the acceptance of several key laws in Dutch history with wide implications. In the penultimate chapter, I have turned to the questions whether and to what extent politicians can enrich themselves, and whether and how the political system enables or mitigates politicians' penchant to pursue their own interests. In this final chapter, I start by briefly recapitulating the most important findings of the preceding chapters (section \ref{sec:recap}). Afterwards, I come back to the data and methods I have used throughout the dissertation, and to the implications for Dutch political history. I also argue that the research findings have broader implications, teaching us about the processes that led to the democratization of Western Europe in the late nineteenth century, and perhaps also more broadly about democratization more generally (section \ref{sec:reflection}). In addition, I argue that my findings call for a more explicit role of political elites, and in particular, intra-elite conflicts, in models that attempt to explain economic growth, franchise extension, or taxation. In tandem, I conclude by outlining several suggestions for further research (section \ref{sec:sfr}). 

%This dissertation started out by asking the question if, how, and to what extent a political elite prioritizes its self-interest, and how this phenomenon interrelates with democratization and a transitioning economy. 



\subsection{Recapitulation of the Main Chapters}\label{sec:recap}
The second chapter of this thesis started out by mapping out the wealth of Dutch politicians over time, in various representative bodies. In doing so, the chapter is supplementing the historical literature about the profile of Dutch politicians in the nineteenth and twentieth centuries \citep{van1983toegang, secker1991ministers, van1999eerste, oomen2020werk} with a perspective that was ruled out before: \cite{van1983toegang} remarked that "the hypothesis that involves background factors having a predictive value for political behavior and political choices especially has turned out to be unprovable so far." This chapter can be seen as bringing new life to this way of thinking, making use of a a publicly available source, the \textit{Memories van Successie}, to get a detailed impression of politicians' personal wealth. This chapter laid bare a particularly large cleavage between the Lower and Upper Houses: whereas the lower house was a diverse place in terms of personal wealth, the upper house was exclusively dominated by extremely wealthy individuals, and this remained so throughout the period. Importantly, the data in the second chapter paved the way for asking analytical questions about the consequences of this wealth profile of the representative bodies, which are the subjects of the third and fourth chapters. 

In a sense, the third and fourth chapter in this dissertation can be interpreted as a test of perspectives by e.g. \cite{tahoun2019personal} and \cite{grossman1996electoral}, who think that politicians have preferences over policies that consist of a (monetary) self-interest component, and a general interest component, which might also include ideology. The empirical question is then what weight a potential self-interest component might have in their decision-making. In the third chapter, my results show that the personal wealth of politicians influences their decision-making: richer politicians are more likely to vote against fiscal legislation than poorer politicians, everything else equal. This analysis exploits the progressivity of fiscal law projects, and finds that politicians who would be hit harder by the acceptance of a new piece of fiscal legislation are systematically less likely to accept the law than politicians who are poorer, conditional on party and a host of other controls. The instrumental variable analyses also provide support for a causal interpretation of these results. 

I also analyze laws in which the effect of acceptance on personal wealth is not so clear \textit{a priori}, specifically, the case of suffrage extension. Additionally, I conduct various placebo tests, and find there is no effect between the acceptance of the law and the personal wealth of politicians, making it unlikely that personal wealth interests actually proxy something else. In sum, this shows that the personal profile of politicians impacted the acceptance of fiscal legislation, and thus, government size and the level of social spending. In terms of the conjectures by \cite{tahoun2019personal} and \cite{grossman1996electoral}, I thus find that the self-interested component is large enough to be detectable statistically. The coefficient on wealth is about five times as small as the coefficient on political party, the most important empirical determinant of voting behavior, allowing me to guess that ideology is about five times more important than self-interest. 

But if self-interest plays an important role in the choices of politicians, why have these laws still been accepted, to the detriment of their own interests? My interpretation makes use of the findings in chapter two and chapter three. In chapter two, I documented the wealth of politicians, and I also focused on the wealth of parliament over time. I observed a pattern of declining parliamentary wealth, which, over the entire period, meant that the median Lower House member was almost 10 times as poor in 1920 than the median Lower House member in 1870. In chapter three, I documented that the speed with which these key laws have been accepted accelerated in the 1900s and 1910s, whereas before, many predecessor law projects with roughly the same goal were rejected by parliament \citep{smit2002omwille, van2013eerste}. I think that there are two main factors underlying these patterns: the first is that certain exogenous economic shocks, which made politicians substantially poorer, have sufficiently mitigated the personal wealth-incentive for these laws to be accepted. Secondly, a part of this effect might have been exacerbated by the franchise extensions we have seen, which increased the diversity of parliament, including in wealth \citep{van1983toegang}. Thus, while there is no evidence of the influence of self-interested behavior in accepting suffrage extensions, these suffrage extensions themselves might have had an indirect effect on mitigating the role of self-interest in politics, through the selection of poorer politicians on average.

This consideration also brings me to the economic relevance of self-interest. The counterfactual scenarios constructed in chapter three imply that, for accepted laws to be rejected, politicians would have had to be richer by a factor of about ten, which is approximately the difference between the wealth of the median Lower House politician in 1870 (before the majority of reforms) and 1920 (afterwards). Conversely, for rejected laws around 1870 to be accepted, politicians would have had to be about 10 times poorer. This implies that the Lower House in 1914, had it been as wealthy as the Lower House in 1870, would likely have rejected the laws they have in fact accepted. Thus, coming back to the quote by \cite{van1983toegang} earlier, I interpret this as evidence of 'background factors', specifically personal wealth, playing a direct role in determining politicians' voting behavior. 

In the fourth chapter, my analysis shows that there are significant returns to politics in the first period, equivalent to approx. a 5 percentage point yearly return on wealth over the remaining lifetime, and also equivalent to about 5-6 a yearly Ministers' salary. The results also show that there are little to no returns to a longer stay in politics, suggesting that the returns to politics are depletable. I rule out various alternative explanations, including the possibility that returns are obtained after a political career, for example, by a lucrative function in finance or in the colonies, or that the results are due to electorates or parties detecting 'bad type' politicians \citep{besley1995does}. 

My interpretation of these results focuses on the nature of the district system \citep{van2018tussen}, promoting close connections between (enfranchised) business leaders and the political representatives of the district, and politicians being able to use their discretion in the lower house to accept law projects that are financially favorable to them, and reject laws that are not. I also provide evidence of the influence of changing institutions on the returns to politics: in particular, I find that these returns are realized in periods when political parties did not exist \citep[cf.][]{eggers2009mps}. This is additional evidence for the supposition that politicians used their discretion to their own financial advantage, hinting again at the existence of a monetary self-interest component in the utility function of politicians: after political parties were founded, party discipline decreased the level of autonomy and discretion of individual politicians \citep{de2001antirevolutionaire,de2014ons}. According to this analysis, political parties, then, are able to discipline politicians enough to make them refrain from engaging in self-interested activities. On the other hand, several other changing institutional changes, such as suffrage extensions, have not been able to influence the returns to politics \citep[see also][]{ashworth2010does}. 

\subsection{Reflection}\label{sec:reflection}

\subsubsection{Dutch Political History}

At places, Dutch political historians have pointed at the possible influence of self-interest in their decision-making, and in the political process in general. For example, \cite{smit2002omwille}, in her study of the process leading to the 1893 income tax, argues that opponents experienced the pressure of the financial elite, while refraining from explicitly involving the personal wealth of politicians. \cite{van1983toegang}, in his study about the background characteristics of Lower House members, admits that information about personal welfare of Lower House members or their family is not taken into account, implicitly acknowledging its potential importance. At the same time, \cite{van2013eerste}, in their history of the first 150 years of Dutch parliamentary politics, do not mention the personal interests of politicians as an important motivating factor, and neither do other similar accounts \cite{de2003het, de2014ons}. \cite{koch2020abraham}, in his biography of Protestant leader Abraham Kuyper, recounts differences in class and manners between the 'man of the people' Kuyper and the dominant aristocrats in parliament, but does not relate that to wealth and self-interest. 

This dissertation, and in particular chapter two, has quantitatively assessed the focus of political historians using several case studies involving key laws in a period of democratization. On the one hand, the analyses show that the key intuitions and analyses in the political history literature are correct: political party and ideological adherence are by far the most important predictors of voting behavior in the lower house, justifying the focus that political historians have applied. On the other hand, I also offer systematic evidence that personal wealth played a role in the political arena, even though the role is smaller than that played by ideology. A general omission of politicians' personal interests from the explanatory model is therefore unwarranted. I also shed light on the influence of personal wealth in the context of democratization: due to the Lower House becoming less wealthy, the incentive to pursue self-interest was mitigated in favor of an ideological choice to accept a broader tax base and higher taxation. While suffrage extension decisions have themselves not been impacted by self-interest, indirectly, the role of suffrage extensions has been to facilitate less wealthy politicians entering the political arena, thereby again mitigating the incentive for politicians to prioritize their self-interest. 

In chapter four, I offer evidence that politicians have been able to enrich themselves due to their political career, offering again direct evidence of the persuasion of self-interest by politicians. This time, the evidence is obtained by comparing the end-of-life wealth of just-elected elected politicians to their nearly-elected counterparts. Many of the analyses I conduct point at the likelihood of a so-called in-office explanation, meaning politicians likely obtain these financial rewards from their behavior while being active of a politician. In short, the contribution of this dissertation is that it offers evidence on the role of self-interest in Dutch politics in two directions. Firstly, I show that personal wealth affects political decision-making in the Lower House, and secondly, I show that politicians use political office for their private benefits. Together, this evidence should strengthen the argument in favor of augmenting the explanation of Dutch democratization with the perspective of self-interested politicians. 


\subsubsection{Political Economy}

In the theoretical political economy literature, politicians are often assumed to be either office-seeking, rent-seeking or partisan politicians \citep{persson2002political}. These models can be both static or dynamic, but in either case, the structure is usually such that politicians can be reelected and obtain electoral benefits, or attempt to seek private benefits, once per period. Focusing on rent-seeking behavior, the findings in chapter 4 challenges this convention in the literature. In particular, I document that the returns to politics can be accrued only in the first period of political office, but not afterwards. I also document that political party discipline is likely to curb the magnitude of the private returns to politics towards zero, implying that rent-seeking might be depletable, giving rise to potentially different dynamics. 

But if there are no private returns to politics in the presence of political parties, what would then serve as motivation for citizens to enter politics? This tension is often present in citizen-candidate models \citep[see e.g.][]{besley2005political}. The findings that I obtain can be reconciled with the framework introduced by \cite{svaleryd2009political}, which might serve as a blueprint for future modeling. In their paper, politicians have two motivations, one of which might be interpreted as a monetary form of extraction, and another represents the benefit of being in office \textit{per se}. The latter can be interpreted as utility that comes paired with having voting power and being able to implement policies closer to one's social preferences and ideology. 

The context and findings of the chapters also illustrate the importance of looking at discord and heterogeneity within the political elite, rather than interpreting the political elite as a group whose interests among themselves are perfectly aligned. In that sense, the results provide support for the approached opted for by \cite{lizzeri2004did} or \cite{llavador2005partisan} rather than by \cite{acemoglu2013political}, even though the approaches are not mutually exclusive. Indeed, the differences between various periods in returns to political office illustrate the need of a dynamic approach often favored by \cite{acemoglu2013political}. This links up with the democratization literature \citep[see e.g.][]{acemoglu2000did, acemoglu2008oligarchic,aidt2019motivates}. Chapter three of this dissertation highlights the interaction between the composition of the parliament, and suffrage extensions and fiscal legislation, to which not much attention is paid in the theoretical literature. 

%What do we learn about democratization in general?

\subsubsection{Methods}

In this dissertation, I have used various causal inference methods to establish relationships between politicians' wealth and their political activity \citep{cunningham2021causal}. In chapter three, I have instrumented endogenous personal wealth of politicians by their arguably exogenous expected inheritance. The price of employing this method was a reduced sample, because data availability of the source that I use limited the number of observations for which data on the instrument was available. In a way, this can be seen as a potential fruitful substitute for the lack of panel data on wealth. Most of the studies which use wealth as an outcome variable \citep{fisman2014private, berg2020politicians, berg2020returns} in a modern setting have opted for panel data. When this is not available, the option that I have pursued seems a fruitful approach. On the other hand, when data on family links is not readily available, this can entail an additional data collection effort. In chapter four, I have used regression discontinuity analysis revolving around close elections to investigate the treatment effects of a political career on personal wealth. In this analysis, I show that covariate balance holds for a large subset of settings, indicating that at the margin, electoral outcomes are likely to be allocated randomly with respect to the potential outcomes. This makes regression discontinuity a good setting for studying the effects of a political career. In my analyses, I have focused throughout on the Lower House and the elections to the Lower House. In principle, this approach could be extended towards indirect elections to the Upper House, and possibly, municipal councils. The principal disadvantage of using regression discontinuity is that it is heavily tied to a particular treatment, in this case, politics. On the other hand, in the period under investigation, electoral districts' and municipalities' boundaries are frequently redrawn. This introduces the possibility of spatial regression discontinuity designs, \citep[e.g.][]{dell2010persistent, egger2015causal, lowes2021concessions}, for example, in researching the effects of belonging to an electoral district with various characteristics in terms of turnout and size. 

\subsubsection{Data Sources}
This dissertation primarily relied on archival sources to collect probate inventories, \textit{Memories van Successie} (MVS), to obtain a reliable measure of politicians' personal wealth \citep{bos1990vermogensbezitters}. Probate inventories have many advantages: they provide a detailed appraisal of a politicians' wealth at the time of decease, and usually, also a detailed inventories consisting of their assets and liabilities, and a separate appraisal of each and every one of them. The completeness of the deceased's wealth had to be declared under oath, and regularly, the tax agency required descendants to file additional declarations of assets that were initially missing. This indicates that a significant amount of time was devoted to ensuring that an individual's full wealth served as the tax base. On the other hand, the MVS also have several disadvantages. For one, it is possible that despite oversight, individuals are still able to hide assets in various ways. To the extent this is done systematically, this potentially biases the results, possibly introducing measurement error or selection bias, or making the estimates less efficient \citep{angrist2008mostly}. Secondly, it provides an overview of an individual's assets at only one point in time, at the end of one's life. In view of life-cycle saving theories in finance, individuals might have various motives to systematically change the composition of their wealth, and anticipate bequests as they get older \citep{dynan2002importance}. 

More broadly, the MVS are available only once for each individual. Research using this data source must then necessarily rely on cross-sectional or cohort data, but cannot use inferential techniques using panel data. In the Netherlands, there exist few possible other sources to obtain a measure of individuals' wealth and income. One alternative source is the \textit{Kohieren van de Gemeentelijke Hoofdelijke Omslag} \citep{klep1987kohieren}, a source detailing municipal taxes paid at the individual-year level. In principle, this source would allow for repeated measurement of income on the basis of taxes paid, although there are a few reservations: the tax base is not the harmonized across municipalities, and the effective tax rate differs from municipality to municipality. Empirical strategies using municipal fixed effects could accommodate this, but various legal changes also complicate that, as municipality-wide average tax base and tax rates change over time. Unfortunately, there exist no systematic archival records of the nation-wide income tax studied in the second chapter of this dissertation. The availability of the so-called \textit{Kohieren van de (Rijks)inkomstenbelasting} is scarce, and highly dependent on the place and time. In specific situations, however, this source can be used for treatments at the micro-level. This source is less suitable for studies like the ones conducted in preceding chapters, however, since they study a population that is geographically spread across the Netherlands. Both of the sources can usually be found in municipal archives (\textit{Stadsarchieven}) rather than provincial archives. 

There are also various other \textit{kohieren}, detailing taxes paid at the individual level related to various asset classes, e.g. real estate or shares. If one wants to study these asset classes specifically, these sources are suitable, but subject to the same limitations as the other \textit{kohieren}. Otherwise, if one wants to study wealth and income as a whole, these sources are heavily bias towards individuals with these specific assets. In sum, I think the MVS are still the most useful to study wealth and income, in terms of coverage, availability, and uniformity. Because of these characteristics, sources like the MVS still have ample opportunities to be used in the future on a scale larger than in this dissertation. Due to advances in deep learning \citep{shen2021layoutparser} and optical character recognition (OCR), I think it is possible to leverage more data and systematically collect, curate and analyze the MVS to study the effects of various interventions in difference-in-difference-like designs. A possible challenge to this process would be the OCR of largely hand-written sources. After about 1900, most of the MVS are typed on a typewrite rather than hand-written, likely facilitating OCR. 

Finally, although the MVS theoretically cover virtually the entire population, in practice, it is sometimes difficult to find specific individuals. In my opinion, this occurs principally because of two reasons. The law stipulates that individuals must file and register the MVS at the registration office managing the place of death. This principle is widely deviated from. For example, it is often difficult to find probate inventories of individuals who have died outside of the Netherlands, because there is no designated office. In addition, descendants of deceased individuals often do not file their declaration at the place of death, but rather, at the office close to the place in which they live, or with which they have a special cultural bonding. In this respect, biographical information about individuals to be found can help locate the likely place of the specific MVS. The second reason why individuals might be difficult to find has to do with archival organization. Oftentimes, individuals' assets are transferred from generation to generation, leading the civil servants administering the probate inventories to use probate inventories from previously deceased parents to investigate the assets of the deceased children. These probate inventories are sometimes not put back, and hence, leaves open a range of possible locations for the parents' probate inventories. In practice, I believe that after having considered the place of death and possibly the place of bonding, it is generally not worth the risk of conducting more search activity for a probate inventory in potentially different archives and places.  

% what about data availability? does it pay to look for more Memories or not? (answer: no)




\subsection{Suggestions for Further Research}\label{sec:sfr}

This dissertation suggests various avenues for further research. 

The findings in chapter three suggest that politicians trade-off self-interest and other factors, among which are ideology and party discipline. In the present-day Dutch context, party discipline has almost become absolute. That might one to suppose that there is no more room for abuse by politicians, and that the problem of politicians pursuing their own interest is anachronistic. Potentially, new research could provide arguments for why that is not the case. Specifically, there might be more subtle ways to prioritize self-interest, for example, by adding amendments and clauses to project laws, or by pressuring political parties into taking up certain points in their electoral program. In a present-day context, analyses of these sources might lead to the discovery of new ways in which politicians can pursue opportunism. 

In addition, party discipline varies significantly across countries. In many countries, party discipline is looser than in the Netherlands. A cross-country analysis could shed light on the generalizability of the findings by relating the extent to which politicians can pursue financial self-interests to the degree of party discipline. 

Furthermore, despite politicians prioritizing their own finances when voting on laws, chapter three confirms that the influence of ideology and party discipline was by far the strongest factor determining their voting behavior \citep[see e.g.][]{de2003het, de2014ons}. In this context, the results in chapter three can be interpreted as a lower bound of the influence of self-interest if self-interests also affect the decision of politicians to join a particular party, whereas the results I obtain are conditional on a political party choice. Based on these considerations, another interesting avenue for future research would suggest that the incentives to joining a political party should also be investigated.

Focusing on the findings in chapter four regarding the private returns to political office, many questions remain. The findings show that politicians can only accrue private returns in the first period of political activity, but not afterwards. Theoretically, many models supposed a static environment, implying that the returns to politics should be constant. In addition, several empirical papers \citep{baltrunaite2020political, bourveau2021political} suggest mechanisms that also imply a constant return curve. Alternative explanations, such as human capital or career paths-focused explanations \citep{eggers2009mps} would imply that the aggregate returns to a political career are larger if one's remaining lifespan is longer. My finding that returns to politics happen only in the first period challenges all of these findings. In further research, it would be interesting to find out to what extent this finding is generalizable. As of yet, many studies focus on only one or a few periods of political office. Theoretically, it would be interesting to rationalize these findings by seeing returns to politics as a depletable resource, where a possible equilibrium would imply the depletion after the first period \citep[cf.][, p. 74]{acemoglu2013political}. Particular attention should be paid to alternative incentives, other than private returns, for citizens to stand as a candidate. 

The findings also suggest that political parties are able to discipline politicians. While there exist several models incorporating party discipline \citep{eguia2011voting, curto2018party}, it is unclear where party discipline comes from. Theoretically, it would be interesting to model party discipline as a product of the interaction between electorates and coalitions of politicians acting under uncertainty. Empirically, it would be interesting to find parallels with different literatures, such as industrial organization. In this way, it might be possible to obtain more precise definitions and measures of party discipline. Finally, the findings hint in various ways at a mechanism encompassing politicians using their voting discretion as a means to obtain private financial advantage. It would be interesting to document a setting in which it is possible to find direct evidence for this conjecture \citep[as in][]{tahoun2019personal}.


\clearpage
\bibliographystyle{apalike}
\bibliography{references}

\end{document}
