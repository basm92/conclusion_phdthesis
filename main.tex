\documentclass[12pt]{extarticle}
\usepackage[utf8]{inputenc}
\usepackage[margin=1in]{geometry}

\usepackage[cmintegrals,cmbraces]{newtxmath}
%\usepackage{ebgaramond}

\usepackage{amsmath}
\usepackage{amsfonts}
%\usepackage{amssymb}

%table numbers within section
\usepackage{chngcntr}
\counterwithin{table}{section}

\usepackage{natbib}
\usepackage{graphicx}
\usepackage{booktabs}
\usepackage{longtable}
\usepackage{caption}

%\frenchspacing
\usepackage{setspace}
\setstretch{1.1}

\usepackage{graphicx}
\usepackage{url}
\usepackage[colorlinks=true, urlcolor = orange, linkcolor=blue, citecolor=purple]{hyperref}

\usepackage{xcolor}
%\usepackage{sectsty}
\usepackage{appendix}

%\subsectionfont{\color{cyan!70!blue}}  % sets colour of chapters
%\sectionfont{\color{blue!60!black}}  % sets colour of sections

%Tables:
\usepackage{dcolumn}
\usepackage{float}
\usepackage{caption}
\usepackage{tabularx}

% Landscape mode
%\usepackage{lscape}
\usepackage{pdflscape}

%Threepart table
\usepackage{threeparttable}
%\usepackage{tabu}

% Modelsummary
\usepackage{booktabs}
\usepackage{siunitx}
\newcolumntype{d}{S[input-symbols = ()]}

% figure notes
\usepackage[capposition=top]{floatrow}

\title{\textbf{PhD Thesis Introduction}}
\author{Bas Machielsen \\ Utrecht University}
\date{\today}

\begin{document}

\maketitle

\section{Concluding Remarks}

In X, Dutch Lower House member and liberal politician Sam van Houten accused his fellow Lower House members of being an oligarchy, purposefully shirking their responsibility of contributing a fair share towards the state finances\footnote{I want to thank Jesper van der Most for bringing this quote to my attention}. His outburst marked the frustration experienced by 


In the third chapter, my results show that the personal wealth of politicians influences their decision-making: richer politicians are more likely to vote against fiscal legislation than poorer politicians, everything else equal. The analyses also provide support for a causal interpretation of these results. In the case of suffrage extension, and various placebo tests, where there is no direct effect between the acceptance of the law and the personal costs politicians bore in case of acceptance, there is no evidence for personal wealth playing a role. This shows that the personal profile of politicians impacted the acceptance of fiscal legislation, and thus, government size and the level of social spending. My interpretation is that certain exogenous economic shocks, which made politicians substantially poorer, have sufficiently mitigated the personal wealth-incentive for these laws to be accepted. 

In the fourth chapter, my analysis shows that there are significant returns to politics in the first period, equivalent to approx. a 5 percentage point yearly return on wealth over the remaining lifetime, and also equivalent to about 5-6 a yearly Ministers' salary. The results also show that there are little to no returns to a longer stay in politics, suggesting that the returns to politics are depletable. I rule out various alternative explanations, including the possibility that returns are obtained after a political career, for example, by a lucrative function in finance or in the colonies. My interpretation of these results focuses on the nature of the district system \citep{van2018tussen}, promoting close connections between (enfranchised) business leaders and the political representatives of the district. I also provide evidence of the influence of changing institutions on the returns to politics: in particular, I find that these returns are realized in periods when political parties did not exist. According to this analysis, political parties, then, are able to discipline politicians enough to make them refrain from engaging in self-interested activities. On the other hand, several other changing institutional changes, such as suffrage extensions, have not been able to influence the returns to politics \citep{ashworth2010does}. 


\section{Suggestions for Further Research}

\end{document}
